\documentclass[10pt]{beamer}

\usetheme{metropolis}
%\usepackage[polish]{babel}
\usepackage{appendixnumberbeamer}
\usepackage{booktabs}
\usepackage[scale=2]{ccicons}
\usepackage{pgfplots}
\usepgfplotslibrary{dateplot}
\usepackage{xspace}

\newcommand{\themename}{\textbf{\textsc{metropolis}}\xspace}

\title{Title}
\subtitle{Subtitle}
\date{\today}
\author{Albert Szadziński}
\institute{University of Silesia}
\titlegraphic{\hfill\includegraphics[height=1.5cm]{logo.pdf}}

\begin{document}
\maketitle

\begin{frame}{Table of contents}
  \setbeamertemplate{section in toc}[sections numbered]
  \tableofcontents[hideallsubsections]
\end{frame}


\section{Introduction}


{
\setbeamertemplate{frame footer}{footer}


\begin{frame}[fragile]{HEADER}

  \alert{href} - \href{https://github.com/hsrmbeamertheme/hsrmbeamertheme}{\textsc{hsrm} Theme} \ldots
	\begin{alertblock}{Potential Problems}
			teststt
	\end{alertblock}
\end{frame}

}




\begin{frame}{Lists}
  \begin{columns}[T,onlytextwidth]
    \column{0.33\textwidth}
      Items
      \begin{itemize}
        \item Milk \item Eggs \item Potatoes
      \end{itemize}

    \column{0.33\textwidth}
      Enumerations
      \begin{enumerate}  \newcounter{density}
    \setcounter{density}{20}
        \item First, \item Second and \item Last.
      \end{enumerate}

    \column{0.33\textwidth}
      Descriptions
      \begin{description}
        \item[PowerPoint] Meeh. \item[Beamer] Yeeeha.
      \end{description}
  \end{columns}
\end{frame}




\begin{frame}{Figures}
  \begin{figure}
  \includegraphics[width=0.3\textwidth,]{logo.pdf}
    \caption{Rotated square from
    \href{http://www.texample.net/tikz/examples/rotated-polygons/}{texample.net}.}
  \end{figure}
\end{frame}


\begin{frame}{Tables}
  \begin{table}

    \begin{tabular}{@{} lr @{}}
      \toprule
      City & Population\\
      \midrule
      Mexico City & 20,116,842\\
      Shanghai & 19,210,000\\
      Peking & 15,796,450\\
      Istanbul & 14,160,467\\
      \bottomrule
    \end{tabular}
        \caption{Largest cities in the world (source: Wikipedia)}
  \end{table}
\end{frame}


\begin{frame}{Blocks}
  Three different block environments
  \begin{columns}[T,onlytextwidth]
    \column{0.5\textwidth}
      \begin{block}{Default}
        Block content.
      \end{block}
      \begin{alertblock}{Alert}
        Block content.
      \end{alertblock}
      \begin{exampleblock}{Example}
        Block content.
      \end{exampleblock}

    \column{0.5\textwidth}
      \metroset{block=fill}
      \begin{block}{Default}
        Block content.
      \end{block}
      \begin{alertblock}{Alert}
        Block content.
      \end{alertblock}
      \begin{exampleblock}{Example}
        Block content.
      \end{exampleblock}
  \end{columns}
\end{frame}

\begin{frame}{Math}
  \begin{equation*}
    e = \lim_{n\to \infty} \left(1 + \frac{1}{n}\right)^n
  \end{equation*}
\end{frame}


\begin{frame}{Line plots}
  \begin{figure}
    \begin{tikzpicture}
      \begin{axis}[
        mlineplot,
        width=0.9\textwidth,
        height=6cm,
      ]

        \addplot+[samples=100] {1/x^2};
        %\addplot+[samples=100] {sin(deg(4*x))};
      \end{axis}
     
    \end{tikzpicture}
  \end{figure}
\end{frame}

\begin{frame}{Bar charts}
  \begin{figure}
    \begin{tikzpicture}
      \begin{axis}[
        mbarplot,
        xlabel={Foo},
        ylabel={Bar},
        width=0.9\textwidth,
        height=6cm,
      ]
      \addplot plot coordinates {(1, 20) (2, 25) (3, 22.4) (4, 12.4)};
      \addplot plot coordinates {(1, 18) (2, 24) (3, 23.5) (4, 13.2)};
      \addplot plot coordinates {(1, 10) (2, 19) (3, 25) (4, 15.2)};
      \legend{lorem, ipsum, dolor}
      \end{axis}
    \end{tikzpicture}
  \end{figure}
\end{frame}




\begin{frame}[standout]
  Backup
\end{frame}


\end{document}
